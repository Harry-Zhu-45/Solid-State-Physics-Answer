\chapter{晶体的结合\label{ch:3}}

因为不考试,本章暂停施工。望周知

% \noindent \textbf{1.\quad} 试证明以等间距排列的一维离子晶体的马德隆常数等于 $2\ln2$。

% \noindent \textbf{证:}

% 设相邻原子间的距离为 $r$, 一个原子的最近邻、次近邻……原子均有 $2$ 个,该晶体的马德隆常数为:

% \begin{align}
%     M &= 2 - \frac{2}{2} + \frac{2}{3} - \frac{2}{4} + \cdots \\
%     &= 2 \left(1-\frac{1}{2}+\frac{1}{3}-\frac{1}{4}+\cdots\right) \\
%     &= 2 \sum_{n=1}^{\infty} (-1)^{n-1} \frac{1}{n} \\
%     &= 2 \ln 2
% \end{align}

% \noindent \textbf{2.\quad} 由实验测得 \ce{NaCl} 晶体的密度为 \qty{2.16}{g/cm^3}, 它的弹性模量为 \qty{2.14e10}{N/m^2}, 试求 \ce{NaCl} 晶体的每对离子内聚能 $\frac{U}{N}$。(已知马德隆常数 $M=1.7476$,\ce{Na} 和 \ce{Cl} 的原子量分别为 $23$ 和 $35.45$)

% \noindent \textbf{解:}

% \ce{NaCl} 晶体中 \ce{Na} 和 \ce{Cl} 的最近距离为 $r_0$,晶胞基矢长为 $2 r_0$,一个晶胞中含有四对正负离子对。所以一个原胞 (一个 \ce{NaCl} 分子) 的体积为:

% \begin{equation}
%     v = 2 r_0^3 = \frac{m}{\rho N} = \frac{(23+35.45) \times 10^{-6}}{2.16 \times 6.02 \times 10^{23}}
% \end{equation}

% 所以 \ce{NaCl} 晶体中的正负离子的平衡间距为:

% \begin{equation}
%     r_0 = \qty{2.82e-8}{cm} = 
% \end{equation}




