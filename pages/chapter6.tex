\chapter{金属电子论\label{ch:6}}

\noindent \textbf{1.\quad}导出一维和二维自由电子气的能态密度。

\noindent \textbf{解:}

一维情形:

自由电子的 Schrödinger 方程为

\begin{equation*}
    -\frac{\hbar^2}{2m} \cdot \frac{\dif^2 \varphi}{\dif x^2} = E \varphi
\end{equation*}

自由电子波函数的解为

\begin{equation*}
    \dif Z = 2 \cdot \frac{L}{2\pi} \cdot 2 \dif k = \frac{2L}{\pi} \dif k = \frac{2L}{\pi} \frac{\sqrt{2m}}{\hbar} \frac{\dif E}{2 \sqrt{E}} = \frac{L}{\pi} \frac{\sqrt{2m}}{\hbar} \frac{\dif E}{\sqrt{E}}
\end{equation*}

且有

\begin{equation*}
    E = \frac{\hbar^2 k^2}{2m}
\end{equation*}

由周期性边界条件 $\varphi(x+L)=\varphi(x)$ ,可得

\begin{equation*}
    k = \frac{2\pi}{L} n
\end{equation*}

在 $k=\sqrt{2mE}/\hbar$ 处于区间 $[k, k+\dif k]$ 时候

由于 $\dif Z = L g_1(E) \dif E$,对比可得

\begin{equation*}
    g_1(E) = \frac{\sqrt{2m}}{\pi\hbar} E^{-\frac{1}{2}}
\end{equation*}

二维情形:

同上,由电子的 Schrödinger 方程:

\begin{equation*}
    -\frac{\hbar^2}{2m} \nabla^2 \varphi = E \varphi
\end{equation*}

得自由电子波函数解:

\begin{equation*}
    \varphi(\mathbf{r}) = \frac{1}{\sqrt{S}} e^{i \mathbf{k} \cdot \mathbf{r}}, \quad S = L^2
\end{equation*}

且

\begin{equation*}
    E(\mathbf{k}) = \frac{\hbar^2 k^2}{2m} = \frac{hbar^2}{2m} (k_x^2 + k_y^2)
\end{equation*}

由周期性边界条件可得

\begin{equation*}
    \begin{cases}
        \varphi(x+L, y) = \varphi(x, y) \\
        \varphi(x, y+L) = \varphi(x, y)
    \end{cases}
\end{equation*}

可得

\begin{equation*}
    k_x = \frac{2\pi}{L} n_x, \quad k_y = \frac{2\pi}{L} n_y
\end{equation*}

在区间 $[k=\sqrt{2mE}/\hbar, k+\dif k]$ 内

\begin{equation*}
    \dif Z = 2 \cdot \frac{S}{(2\pi)^2} \dif \mathbf{k} = \frac{S}{2\pi^2} \cdot 2\pi k \dif k = \frac{mS}{\pi \hbar^2} \dif E
\end{equation*}

同理,与 $\dif Z=S g_2(E) \dif E$ 比较可得

\begin{equation*}
    g_2(E) = \frac{m}{\pi \hbar}
\end{equation*}

\noindent \textbf{2.\quad} 若二维电子气的面密度为 $n_s$,证明它的化学势为:

\begin{equation*}
    \mu = k_B T \ln \left[\exp\left(\frac{\pi\hbar^2 n_s}{m k_B T}\right)-1\right]
\end{equation*}

\noindent \textbf{解:}

由前一题已经求得能态密度

\begin{equation*}
    g(E) = \frac{m}{\pi\hbar}
\end{equation*}

电子气体的化学势 $\mu$ 由下式决定:

\begin{equation*}
    N = \int_0^\infty g(E) L^2 \dif E = \frac{L^2}{\pi \hbar^2} \int_0^\infty \frac{\dif E}{e^{(E-\mu)/k_B T}+1}
\end{equation*}

我们令 $x\equiv(E-\mu)/k_B T$,注意到 $n_s=\frac{N}{L^2}$

\begin{align*}
    n_s &= \frac{k_B T m}{\pi\hbar^2} \int_{-\mu/k_B T}^{\infty} \frac{1}{e^x+1} \dif x \\
    &= \frac{k_B T m}{\pi\hbar^2} \int_{-\mu/k_B T}^{\infty} \frac{\dif e^x}{e^x(e^x+1)} \\
    &= \frac{k_B T m}{\pi\hbar^2} \left. \ln\frac{e^x}{e^x+1} \right|_{-\mu/k_B T}^{\infty} \\
    &= \frac{k_B T m}{\pi\hbar^2} \ln \left(e^{\mu/k_B T}+1\right)
\end{align*}

那么可以求出 $\mu$

\begin{equation*}
    \mu = k_B T \ln \left[\exp\left(\frac{\pi\hbar^2 n_s}{m k_B T}\right)-1\right]
\end{equation*}

\noindent \textbf{3.\quad} \ce{^3 He} 是费米子,液体 \ce{^3 He} 在绝对零度附近的密度为 \qty{0.081}{g.cm^{-3}}。计算它的费米能压和费米温度 $T_F$。

\noindent \textbf{解:}

\ce{^3 He} 的数密度:

\begin{equation*}
    n = \frac{N}{V} = \frac{N}{M} \cdot \frac{M}{V} = \rho \cdot \frac{N}{M} = \frac{\rho}{m}
\end{equation*}

其中 $m$ 为单个 \ce{^3 He} 粒子的质量

\begin{equation*}
    k_F = (3\pi^2 n)^{\frac{1}{3}} = \left(\frac{6\pi^2 \rho}{m}\right)^{\frac{1}{3}}
\end{equation*}

可得

\begin{equation*}
    E_F = \frac{\hbar^2 k^2}{2m} = \frac{\hbar^2}{2m} \left(\frac{6\pi^2 \rho}{m}\right)^{\frac{2}{3}}
\end{equation*}

代入数据,可以算得:

\begin{equation*}
    E_F = \qty{6.8577e-23}{J} = \qty{4.28e-4}{eV}
\end{equation*}

则:

\begin{equation*}
    T_F = \frac{E_F}{k_B} = \qty{4.79}{K}
\end{equation*}

\noindent \textbf{4.\quad} 金属钾在低温下的摩尔电子比热的实验值为:$c_e= 2.08 T \unit{mJ/(mol.K)}$, 试用自由电子气模型求它的费米能扁及状态密度 $g(E_F)$

\noindent \textbf{解:}

考虑费米球模型,在费米面以内的粒子吸收能量跃出费米面的数目期望是:

\begin{equation*}
    \overline{N} = c \int_{E_F-\frac{3}{2}kT}^{\infty} E^{1/2} \dif E = \frac{9}{4} N \frac{kT}{E_F}
\end{equation*}

这些粒子共吸收能量:

\begin{equation*}
    \overline{E} = \frac{\overline{N} \cdot \frac{3}{2}kT}{N} = \frac{27}{4} \cdot \frac{k_B^2 T^2}{E_F}
\end{equation*}

则相应的热容量为:

\begin{equation*}
    C_{ve} = \frac{\partial \overline{E}}{\partial E} = \frac{27}{4} \cdot \frac{k_B^2 T}{E_F} = \lambda T
\end{equation*}

其中:$\lambda=\frac{27}{4} \cdot \frac{k_B^2}{E_F}$

由题设数据,代入上式,可求出 $E_F$ 及 $g(E_F)$:

\begin{align*}
    E_F &= \frac{27 k_B^2 N_A}{4} = \qty{2.235e-3}{eV} \\
    g(E_F) &= \frac{3n V_N}{2 E_F} = \frac{3 N_A}{2 E_F} = \num{2.425e45}
\end{align*}

\noindent \textbf{5.\quad} 银是一价金属,在 $T=\qty{295}{K}$ 时,银的电阻率 $P=\qty{1.61e-6}{\Omega.cm}$,在 $T=\qty{20}{K}$ 时,电阻率 $\rho=\qty{0.038e-9}{\Omega.cm}$。求在低温和室温时电子的自由程。银的原子量为 $107.87$,密度为 \qty{10.5}{g/cm^3}。

\noindent \textbf{解:}

由

\begin{equation*}
    \rho = \frac{1}{\sigma} = \frac{m V_F}{n e^2 l}
\end{equation*}

可得

\begin{equation*}
    l = \frac{m V_F}{n e^2 \rho}
\end{equation*}

又因为

\begin{equation*}
    n = \frac{N}{V} = \frac{N}{M} \frac{M}{V} = \rho_0 \cdot \frac{N}{N_A} \cdot \frac{N_A}{M} = \frac{\rho_0 \cdot N_A}{M_s}
\end{equation*}

其中 $N_A$ 为阿伏加德罗常数,$M_s$ 为 \ce{Ag} 的原子溢,$\rho_0$ 为 \ce{Ag} 的密度。将上式代入 $l$ 的表达式,并代入数据可得:

当 $T=\qty{295}{K}$ 时,$l=\qty{3.7e-4}{m}$,

当 $T=\qty{20}{K}$ 时,$l=\qty{1.6}{m}$,

在计算过程中,已取 $V_F=\qty{e6}{m}$。

\noindent \textbf{6.\quad} Hunter S.C 和 F.R.N.Nabarro 曾计算铜中每厘米位错线引起的电阻率如下:

\begin{enumerate}
    \item 刃型位错 $\Delta_{\rho E} = \qty{0.59e-20}{\Omega.cm}$
    \item 螺型位错 $\Delta_{\rho s} = \qty{0.18e-20}{\Omega.cm}$
\end{enumerate}

假定刃型位错和螺型位错有相同的密度 (位错密度为 \qty{1}{cm^2} 有多少条位错线)。已知位错产生的电阻率 $\Delta \rho = \qty{2e-8}{\Omega.cm}$,问铜中的位错密度是多少?

\noindent \textbf{解:}

设密度为 $X$,由题意可以列出方程:

\begin{equation*}
    \left(\Delta_{\rho E} + \Delta_{\rho E}\right) \cdot x = \Delta \rho
\end{equation*}

\noindent \textbf{7.\quad} 在室温下金属铍的霍尔系数为 \qty{2.44e-10}{m^3.C^{-1}},求铍中空穴密度。

\noindent \textbf{解:}

由霍尔系数定义 $R_H=\frac{1}{pe}$ 得

\begin{equation*}
    p = \frac{1}{e \cdot R_H} = \qty{2.56e28}{m^{-3}}
\end{equation*}

\noindent \textbf{8.\quad} 试计算 \ce{Cs} 在 $T=\qty{1000}{K}$ 时热电子发射的电流密度。

\noindent \textbf{解:}

电子热发射的电流密度函数为:

\begin{equation*}
    j = 4\pi e \left[\frac{m(kT)^2}{\hbar^3}\right] e^{-\frac{\phi}{kT}}
\end{equation*}

查表可得 \ce{Cs} 的功函数为 \qty{1.81}{eV}。代入数据到上式中可以算得:

\begin{equation*}
    j = \qty{9.2e2}{A.m^{-2}}
\end{equation*}

\noindent \textbf{9.\quad} \ce{Al} 等离子体能量 $\hbar \omega_p$ 的实验值为 \qty{15.3}{eV},按照自由电子气模型的电子密度为 $n=\qty{18.06e22}{m^{-3}}$, 求 $\hbar \omega_p$ 的理论值。

\noindent \textbf{解:}

由等离子体振荡频率关系式:

\begin{equation*}
    \omega_p^2 = \frac{n e^2}{\varepsilon_0 m}
\end{equation*}

故:

\begin{equation*}
    \hbar \omega_p = \hbar e \sqrt{\frac{n}{\varepsilon_0 m}} = \qty{15.7}{eV}
\end{equation*}
